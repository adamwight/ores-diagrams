% Data flow diagram
% Template adapted from http://www.texample.net/tikz/examples/data-flow-diagram/ by David Fokkema
\documentclass{article}
\usepackage{tikz}
\usetikzlibrary{arrows,shapes}

% Custom, should be provided in the same directory.
\usepackage{datastore}

\begin{document}
\begin{tikzpicture}[
  font=\sffamily,
  every matrix/.style={ampersand replacement=\&,column sep=2cm,row sep=2cm},
  interface/.style={draw,thick,regular polygon,regular polygon sides=4,inner sep=0},
  process/.style={draw,thick,rounded corners,inner sep=.3 cm},
  datastore/.style={draw,thick,shape=datastore,inner sep=.3cm},
  to/.style={->,>=stealth',shorten >=1pt,semithick,font=\sffamily\footnotesize},
  every node/.style={align=center}]

  % Position the nodes using a matrix layout
  \matrix{
        \& \node[interface] (consumer) {Consumer}; \\
      \node[datastore] (cache) {Score cache};
        \& \node[process] (api) {ORES API}; \\
        \& \node[process] (celery) {Task queue};
          \& \node[process] (precache) {Precaching \\ daemon}; \\
      \node[datastore] (model) {Model};
        \& \node[process] (score) {Calculate score};
          \& \node[interface] (mediawiki) {MediaWiki \\ API}; \\
        \& \node[process] (extract) {Extract features}; \\
  };

  % Draw the arrows between the nodes and label them.
  \draw[to] (api) -- (consumer);
  \draw[to] (cache) --  node[midway, above] {cached result} (api);
  \draw[to] (api) to[bend right=25] node[midway, left] {job} (celery);
  \draw[to] (celery) to[bend right=25] node[midway, right] {response} (api);
  \draw[to] (precache) -- node[midway, above] {job} (celery);
  \draw[to] (score) -- node[midway, left] {score} (celery);
  \draw[to] (score) -- (cache);
  \draw[to] (model) -- (score);
  \draw[to] (extract) -- (score);
  \draw[to] (mediawiki) -- node[midway, right] {recent changes} (precache);
  \draw[to] (mediawiki) -- node[midway, right] {revision content \\ revision metadata} (extract);
\end{tikzpicture}
\end{document}
